\chapter{Introduction}

\section{Research Overview}

The University's provided Extenuating Forms System (ECF) is currently based on pen and paper. This is incredibly cumbersome to both staff and students creating a variety of problems from students being unable to track where in the process their form is currently, to staff losing documents or having to track down lost emails. Students are responsible for locating their own head of department and ensuring that a new form is emailed to the correct person as a Microsoft Word document, all while being potentially ill or suffering from whatever issue caused the need for the extenuating circumstance in the first place. All these issues compound into common occurrences of rejections from typos, to missing small parts of the form given the quantity of information being asked, without any verification of form approval until potential weeks of waiting for a response.
\newline
\newline
This project will explore the development of a web-based application to create a portal for students to give early feedback on a demo; utilising that gained knowledge from early users via a survey to help with future works as well as measuring demands for such a system. This project will look at different toolsets offered to developers to help with the creation of this system as well as architecting a maintainable codebase for future use that will last for many years to the needs and demands of the University. This project will utilise a variety of frameworks and toolsets including PostgresQL for data storage, React for providing a small UI library and UI elements, as well as Django for provinding back-end calls and handling functional actions by a user. By developing the demo in these frameworks, it makes it easy for future maintainers to be able to fork the project and further research in the future.

\pagebreak

\section{Aims and Objectives}

The aim of this project is to investigate and develop a working prototype for users to assess necessary features and desires of an online ECF system. This work is for looking at ease of use as well as implementing modern frameworks to simplify maintainability down the line as well as making it easy to update UI components for other maintainers. The implemented system should be accessible to a variety of technical and nontechnical users providing the most seamless experience possible. The experience of using the demo will then be compared by users to the current system provided by the university to assess whether the resource cost is worthwhile to further development as well as any user positives to a paper-based system that is not present with the online demo. 

\section{Overview of the Report}

The Report consists of the following chapters:
\begin{itemize}
  \item Chapter 2: Details the larger body of frameworks and toolsets used to construct modern web applications.
  \item Chapter 3: Details the design and implementation of a demo that will be used to collect invaluable feedback from user experiences.
  \item Chapter 4: Describes implementation technique and design choices within the codebase of the demo.
  \item Chapter 5: Analyzes the survey results and feedback.
  \item Chapter 6: Breaks apart direct user feedback and potential issues with code scaling.
  \item Chapter 7: Develops a conclusion and most demanded future work from survey results.
\end{itemize}