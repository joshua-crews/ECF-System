\chapter{Conclusion}

\section{Project Summary}

The aim of this project was to set out to develop a working proof of concept demo to provide users an experience of creating an extenuating form online in a far easier and logical way. This would solve the problem of how the current system provided by the university is completely cumbersome and not user friendly at all, along with the fact that users will already be experiencing hardships further compounding the difficulty of completing and submitting an extenuating form.
\newline
\newline
There were multiple issues with the system that were found upon deployment as serving data across the web creates a whole set of problems that could not be easily prepared for from a local machine. Atop this, there also was a set of user experience issues that were not anticipated to be issues early in development that culminated in the need to provide a better experience after initial release to the public. Ultimately having gained great insight into how demos are approached, no user was ever able to hack into another user’s account nor did the system ever crash, resulting in unparalleled uptime that made it so that users could respond to the survey whenever they wanted. This also made it so that the project became an incredible development experience for stress testing what could go wrong from unforeseen user demands. The survey provided further insight into future development that could be made and what would be an utmost priority for moving a system from a demo to a full tool utilised by the university.

\section{Future Work}

It was incredibly satisfactory in the end product to have deployed the demo and have learned all the different perspectives of creating a web app. Have the ability to write the instructions for frontend work, backend work, learn the side of the product team to collect user statistics, as well as understand the constraints of devops and deploy a local application to the web. Gaining insight into all of these problems were incredible learning opportunities and provided invaluable data to the progression of the demo as well as its own development.
\newline
\newline
Mobile support, while thought originally to be a priority, turned out to not have a high demand and thus little emphasis should be placed on it moving forward. The bigger piece for users is providing more context as to what is going wrong when a form is filled out incorrectly as well as giving the user more information about what was filled out incorrectly. This, while patched in parts of the application, is still a critical feature that many reported would be incredibly useful. This includes also developing the option for a user to view the entire form in one page and submit it in one go. It was noted from users the desire for page numbers and the option to view all data before submitting. While it was initially thought that in development a set of small pages would be easier for one to read, it was realised not all users desire this level of hand-holding by the page itself. To patch this, a setting would be implemented for users to switch between different form presentation modes.
\newline
\newline
Another point of future work is the development of more support for staff. System staff should be able to log in with their own credentials, see a list of ongoing forms that need reviewing, and as such be able to go through each one. This work was not implemented in the demo as this was feedback driven by students, however the system is built to easily accommodate staff requirements with little work needed down the line. Given the componentized nature of the frontend as well, little work would be needed in the design department in order to update the templates for reviewers or staff.
\newline
\newline
Developing all the technical aspects of scalability mentioned in section 6.3 is another core piece of development that was missing. While an implementation of the system stands that it would only need to serve a maximum size of the university student body, this ability to scale with the university growing with the student body automates most of the work for IT and would not need thought or much maintenance for many years. Streamlining this would be a logical step after completing many of the requested user experience features.