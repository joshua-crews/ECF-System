\chapter{Requirements and Analysis}

\section{Introduction}

In this section of the report, there will be analysis of the overall system architecture and required design elements in order to create a short demo to collect user feedback. A sample user will be able to log into a system from their own computer and be able to complete a crude extenuating form. Then the user can provide feedback on their experience when compared to the system provided by the university in its current state. This chapter will also discuss the requirements for the demo and the specifics of the tools that will be used to both develop and deploy the new system for access from a wider audience.

\section{Project Internal Structure}

For this experiment, as discussed in section 2.2 there will need to be both a back-end system for handling user data and securely manipulating it along with a front-end interactable system that makes the user able to have an easy experience when interacting with the website. This will require creating endpoints for the frameworks to interact between and allow the sharing of data provided by the user. The choice was made to implement the system using a REST API and not GraphQL as it is a new framework and while studies are showing that it is incredibly easy to master after a short period of time, the window for developing the demo and then also deploying it is so short that it would be better to maximise the amount of time learning on deployment rather than internal project structure. There is also the note that moving forward it would be an easy migration to GraphQL from a simple REST architecture, so that will be discussed later in Chapter 6.
\newline
\newline
The project will also be run completely on one computer instance with the possibility for scalability to run the front-end framework on one AWS instance and the back-end on a second instance, however due to project resource limitations, only one computer will be used to run both pieces of software to both simplify deployment and configuration. Nginx will also be used to manage incoming and outgoing HTTP traffic on the remote server which is not required for local development. This is to handle all requests made on port 80 which will be designated for http requests and handling allowed CORS (Cross Origin Resource Sharing) data handling. All this setup should make the project durable for field testing with a user survey that will collect invaluable information about user experiences and what is most desired working forward from the demo.

\section{Back-end System Design Choice}

The back-end code will be constructed using Django as it provides all the tools necessary for rapid development as well as a massive set of security packages that make it easier for web deployment. Django also comes prepackaged with a library for interfacing directly with Postgres which will make for a strong database choice. This includes features such as SQL injection protection as well as the ability to create and manipulate models directly within python rather than SQL statements. The other advantage is that Django provides SQLite for testing purposes as if the decision is made for unit or integration testing down the line there is a robust and solid framework that requires little work on a docker container configuration each time a series of tests are run.
\newline
\newline
Another advantage to choosing to use Django is that due to the framework’s well established usage, there will be infrequent updates with incredibly large amounts of testing already done so that as development is completed there will be little need to worry about platform instability. Using a popular framework provides all the tools and support of the greater open-source community without having to worry about creating libraries to do things that other developers have already done, thus saving time for this research. Django’s provided REST framework library takes this a step further by creating easy to understand GUIs during development and an easy to implement swagger system while maintaining all the expected HTTP response types and expected outputs. This will later be demonstrated in chapter 5 when the demo is launched to users who attempt to backdoor into the system only to find that Django has many of these basic security systems thought out in advance.
\newline
\newline
Postgres was chosen as the database management system due to the long time support of the open-sourced platform but also due to the strength of using SQL to avoid the potential pitfalls of document driven databases discussed in section 2.6. While a document driven database does allow the flexibility of even higher scalability, this really isn't an issue for a project of this size nor the fact that a deployed version of this code would ever have that scale of traffic given the size of the university population as a whole. Having a database needing to manage only a few hundred thousand forms compared to the scale of billions that document driven databases are designed for simply isn’t an issue, especially in a demo \cite{Li_Manoharan_2013}.

\section{React Framework with Vite}

All front-end work and structure will be done using the React library. The reason this choice was made was that by using the Vite build pipeline, the project is offered the opportunity to use Typescript and JSX without having to compromise on additional libraries that can be installed if needed. This makes it so that development can be fast using the toolsets of React without needing to worry about actions such as DOM hydration and the strict rendering mode provided by React. Vite also generates JEST unit tests right into the project on startup making it so that there is little need to worry about linting or unit tests down the road if they are used in constructing a deployment pipeline.
\newline
\newline
The reason Angular was not chosen for the frontend framework is because while Angular is incredibly powerful framework, it comes with a variety of tools that simply aren't needed and would add to the final built project size, thus lowering the loading speeds for a client. While having access to tools like dependency injection and data binding are amazing for large scale applications, there simply isn't a need for a simplistic system such as the extenuating form system that really submits a large form to the backend and then retrieves a list of those forms back to a user. React being a simple UI library makes it far more lightweight especially if the machine running both services is not incredibly powerful.

\section{Nginx Deployment with Gunicorn}

In order to collect survey results from the users, there needs to be a hosted service of the demo provided to users so that non-technical responders have the ability to understand how to connect and use the system, as it would be nearly impossible to ask a person to download the project and configure it to run on their own machine. In order to set up a machine with domain routing and security of the API endpoints as well as serving any static media, Nginx provides all the tools needed to configure an environment as well as simplicity for handling requests to pass it to the backend.
\newline
\newline
In order to protect the backend server and provide a WSGI server, Gunicorn interfaces perfectly with Django and the python environment as a whole. It is completely available within pip so there is little need for setup in a production environment but also it provides the ability to write configuration files with python. This makes it so that I can dedicate as many necessary virtual threads to running the backend service without overloading the computer running the website. This makes it so that the website can handle multiple requests concurrently but also keep multiple processes running without overlapping or causing any race conditions within the database. This follows in line with PEP3333 set by the Python Foundation while keeping the application lightweight and fast.
\newline
\newline
As no domain name will need to be purchased for the purposes of this demo as it will be taken down after research is complete, a free service called DuckDNS will be used. This provides a quick a short domain name that is easily able to be typed and remembered by a user rather than typing in a server address of pure numbers. The only drawback to this is that it is nearly impossible to issue an SSL certificate for secure website access, but as this is a demo it should not be a concern at any point.